\documentclass[]{article}
\usepackage{lmodern}
\usepackage{amssymb,amsmath}
\usepackage{ifxetex,ifluatex}
\usepackage{fixltx2e} % provides \textsubscript
\ifnum 0\ifxetex 1\fi\ifluatex 1\fi=0 % if pdftex
  \usepackage[T1]{fontenc}
  \usepackage[utf8]{inputenc}
\else % if luatex or xelatex
  \ifxetex
    \usepackage{mathspec}
  \else
    \usepackage{fontspec}
  \fi
  \defaultfontfeatures{Ligatures=TeX,Scale=MatchLowercase}
\fi
% use upquote if available, for straight quotes in verbatim environments
\IfFileExists{upquote.sty}{\usepackage{upquote}}{}
% use microtype if available
\IfFileExists{microtype.sty}{%
\usepackage{microtype}
\UseMicrotypeSet[protrusion]{basicmath} % disable protrusion for tt fonts
}{}
\usepackage[margin=1in]{geometry}
\usepackage{hyperref}
\hypersetup{unicode=true,
            pdftitle={writeup},
            pdfborder={0 0 0},
            breaklinks=true}
\urlstyle{same}  % don't use monospace font for urls
\usepackage{color}
\usepackage{fancyvrb}
\newcommand{\VerbBar}{|}
\newcommand{\VERB}{\Verb[commandchars=\\\{\}]}
\DefineVerbatimEnvironment{Highlighting}{Verbatim}{commandchars=\\\{\}}
% Add ',fontsize=\small' for more characters per line
\usepackage{framed}
\definecolor{shadecolor}{RGB}{248,248,248}
\newenvironment{Shaded}{\begin{snugshade}}{\end{snugshade}}
\newcommand{\AlertTok}[1]{\textcolor[rgb]{0.94,0.16,0.16}{#1}}
\newcommand{\AnnotationTok}[1]{\textcolor[rgb]{0.56,0.35,0.01}{\textbf{\textit{#1}}}}
\newcommand{\AttributeTok}[1]{\textcolor[rgb]{0.77,0.63,0.00}{#1}}
\newcommand{\BaseNTok}[1]{\textcolor[rgb]{0.00,0.00,0.81}{#1}}
\newcommand{\BuiltInTok}[1]{#1}
\newcommand{\CharTok}[1]{\textcolor[rgb]{0.31,0.60,0.02}{#1}}
\newcommand{\CommentTok}[1]{\textcolor[rgb]{0.56,0.35,0.01}{\textit{#1}}}
\newcommand{\CommentVarTok}[1]{\textcolor[rgb]{0.56,0.35,0.01}{\textbf{\textit{#1}}}}
\newcommand{\ConstantTok}[1]{\textcolor[rgb]{0.00,0.00,0.00}{#1}}
\newcommand{\ControlFlowTok}[1]{\textcolor[rgb]{0.13,0.29,0.53}{\textbf{#1}}}
\newcommand{\DataTypeTok}[1]{\textcolor[rgb]{0.13,0.29,0.53}{#1}}
\newcommand{\DecValTok}[1]{\textcolor[rgb]{0.00,0.00,0.81}{#1}}
\newcommand{\DocumentationTok}[1]{\textcolor[rgb]{0.56,0.35,0.01}{\textbf{\textit{#1}}}}
\newcommand{\ErrorTok}[1]{\textcolor[rgb]{0.64,0.00,0.00}{\textbf{#1}}}
\newcommand{\ExtensionTok}[1]{#1}
\newcommand{\FloatTok}[1]{\textcolor[rgb]{0.00,0.00,0.81}{#1}}
\newcommand{\FunctionTok}[1]{\textcolor[rgb]{0.00,0.00,0.00}{#1}}
\newcommand{\ImportTok}[1]{#1}
\newcommand{\InformationTok}[1]{\textcolor[rgb]{0.56,0.35,0.01}{\textbf{\textit{#1}}}}
\newcommand{\KeywordTok}[1]{\textcolor[rgb]{0.13,0.29,0.53}{\textbf{#1}}}
\newcommand{\NormalTok}[1]{#1}
\newcommand{\OperatorTok}[1]{\textcolor[rgb]{0.81,0.36,0.00}{\textbf{#1}}}
\newcommand{\OtherTok}[1]{\textcolor[rgb]{0.56,0.35,0.01}{#1}}
\newcommand{\PreprocessorTok}[1]{\textcolor[rgb]{0.56,0.35,0.01}{\textit{#1}}}
\newcommand{\RegionMarkerTok}[1]{#1}
\newcommand{\SpecialCharTok}[1]{\textcolor[rgb]{0.00,0.00,0.00}{#1}}
\newcommand{\SpecialStringTok}[1]{\textcolor[rgb]{0.31,0.60,0.02}{#1}}
\newcommand{\StringTok}[1]{\textcolor[rgb]{0.31,0.60,0.02}{#1}}
\newcommand{\VariableTok}[1]{\textcolor[rgb]{0.00,0.00,0.00}{#1}}
\newcommand{\VerbatimStringTok}[1]{\textcolor[rgb]{0.31,0.60,0.02}{#1}}
\newcommand{\WarningTok}[1]{\textcolor[rgb]{0.56,0.35,0.01}{\textbf{\textit{#1}}}}
\usepackage{graphicx,grffile}
\makeatletter
\def\maxwidth{\ifdim\Gin@nat@width>\linewidth\linewidth\else\Gin@nat@width\fi}
\def\maxheight{\ifdim\Gin@nat@height>\textheight\textheight\else\Gin@nat@height\fi}
\makeatother
% Scale images if necessary, so that they will not overflow the page
% margins by default, and it is still possible to overwrite the defaults
% using explicit options in \includegraphics[width, height, ...]{}
\setkeys{Gin}{width=\maxwidth,height=\maxheight,keepaspectratio}
\IfFileExists{parskip.sty}{%
\usepackage{parskip}
}{% else
\setlength{\parindent}{0pt}
\setlength{\parskip}{6pt plus 2pt minus 1pt}
}
\setlength{\emergencystretch}{3em}  % prevent overfull lines
\providecommand{\tightlist}{%
  \setlength{\itemsep}{0pt}\setlength{\parskip}{0pt}}
\setcounter{secnumdepth}{0}
% Redefines (sub)paragraphs to behave more like sections
\ifx\paragraph\undefined\else
\let\oldparagraph\paragraph
\renewcommand{\paragraph}[1]{\oldparagraph{#1}\mbox{}}
\fi
\ifx\subparagraph\undefined\else
\let\oldsubparagraph\subparagraph
\renewcommand{\subparagraph}[1]{\oldsubparagraph{#1}\mbox{}}
\fi

%%% Use protect on footnotes to avoid problems with footnotes in titles
\let\rmarkdownfootnote\footnote%
\def\footnote{\protect\rmarkdownfootnote}

%%% Change title format to be more compact
\usepackage{titling}

% Create subtitle command for use in maketitle
\providecommand{\subtitle}[1]{
  \posttitle{
    \begin{center}\large#1\end{center}
    }
}

\setlength{\droptitle}{-2em}

  \title{writeup}
    \pretitle{\vspace{\droptitle}\centering\huge}
  \posttitle{\par}
    \author{}
    \preauthor{}\postauthor{}
    \date{}
    \predate{}\postdate{}
  

\begin{document}
\maketitle

\hypertarget{data-science-5620-deliverable-08}{%
\subsection{Data Science 5620 --- Deliverable
08}\label{data-science-5620-deliverable-08}}

\hypertarget{coverge-probability}{%
\subsubsection{Coverge Probability}\label{coverge-probability}}

\hypertarget{rastko-stojsin}{%
\section{\#\#\# Rastko Stojsin}\label{rastko-stojsin}}

Coverage probability is an important operating characteristic of methods
for constructing interval estimates, particularly confidence intervals.

\textbf{Definition:} For the purposes of this deliverable, define the
95\% confidence interval of the mean to be the middle 95\% of sampling
distribution of the mean. Similarly, the 95\% confidence interval of the
median, standard deviation, etc. is the middle 95\% of the respective
sampling distribution.

\textbf{Definition:} For the purposes of this deliverable, define the
coverage probability as the long run proportion of intervals that
capture the population parameter of interest. Conceptualy, one can
calculate the coverage probability with the following steps

\begin{enumerate}
\def\labelenumi{\arabic{enumi}.}
\tightlist
\item
  generate a sample of size \emph{N} from a known distribution
\item
  construct a confidence interval
\item
  determine if the confidence captures the population parameter
\item
  Repeat steps (1) - (3) many times. Estimate the coverage probability
  as the proportion of samples for which the confidence interval
  captured the population parameter.
\end{enumerate}

\hypertarget{suggested-steps}{%
\subsection{Suggested steps}\label{suggested-steps}}

\begin{Shaded}
\begin{Highlighting}[]
\CommentTok{# set sample, pop, and conf int params}
\NormalTok{sample.size =}\StringTok{ }\DecValTok{201}
\NormalTok{pop.mean =}\StringTok{ }\DecValTok{0}
\NormalTok{pop.sd =}\StringTok{ }\DecValTok{1}
\NormalTok{conf.int =}\StringTok{ }\FloatTok{0.95}
\CommentTok{# set sim params}
\NormalTok{sims.sampledist <-}\StringTok{ }\DecValTok{2000}
\NormalTok{sims.evaluation <-}\StringTok{ }\DecValTok{5000}
\CommentTok{# reset counters}
\NormalTok{captured.medi <-}\StringTok{ }\DecValTok{0} 
\NormalTok{pop.medi <-}\StringTok{ }\DecValTok{0}
\NormalTok{cis <-}\StringTok{ }\KeywordTok{array}\NormalTok{(}\OtherTok{NA}\NormalTok{, }\KeywordTok{c}\NormalTok{(sims.evaluation, }\DecValTok{2}\NormalTok{))}
  
\CommentTok{# begin outer look to evaluate simulations}
\ControlFlowTok{for}\NormalTok{ (i }\ControlFlowTok{in} \DecValTok{1}\OperatorTok{:}\NormalTok{sims.evaluation) \{}
  \CommentTok{# find rnorm from pop and sample size}
\NormalTok{  data <-}\StringTok{ }\KeywordTok{rnorm}\NormalTok{(}\DataTypeTok{n=}\NormalTok{sample.size, }\DataTypeTok{mean=}\NormalTok{ pop.mean, }\DataTypeTok{sd =}\NormalTok{ pop.sd)}
\NormalTok{  xbar <-}\StringTok{ }\KeywordTok{mean}\NormalTok{(data)}
\NormalTok{  s <-}\StringTok{ }\KeywordTok{sd}\NormalTok{(data)}
\NormalTok{  medi.star <-}\StringTok{ }\KeywordTok{rep}\NormalTok{(}\OtherTok{NA}\NormalTok{, sims.sampledist)}
  
  \CommentTok{# inner loop}
  \ControlFlowTok{for}\NormalTok{ (j }\ControlFlowTok{in} \DecValTok{1}\OperatorTok{:}\NormalTok{sims.sampledist) \{}
\NormalTok{    data.star =}\StringTok{ }\KeywordTok{rnorm}\NormalTok{(sample.size, xbar, s)}
\NormalTok{    medi.star[j] <-}\StringTok{ }\KeywordTok{median}\NormalTok{(data.star)}
\NormalTok{  \}}
  
\NormalTok{  ci.medi <-}\StringTok{ }\KeywordTok{quantile}\NormalTok{(medi.star, }\KeywordTok{c}\NormalTok{((}\DecValTok{1}\OperatorTok{-}\NormalTok{conf.int)}\OperatorTok{/}\DecValTok{2}\NormalTok{,}\DecValTok{1}\OperatorTok{-}\NormalTok{(}\DecValTok{1}\OperatorTok{-}\NormalTok{conf.int)}\OperatorTok{/}\DecValTok{2}\NormalTok{))}
\NormalTok{  cis[i, ] <-}\StringTok{ }\NormalTok{ci.medi}
  
\NormalTok{  check.medi <-}\StringTok{ }\NormalTok{(ci.medi[}\DecValTok{1}\NormalTok{] }\OperatorTok{<=}\StringTok{ }\NormalTok{pop.medi) }\OperatorTok{&}\StringTok{ }\NormalTok{(pop.medi }\OperatorTok{<=}\NormalTok{ci.medi[}\DecValTok{2}\NormalTok{])}
\NormalTok{  captured.medi <-}\StringTok{ }\NormalTok{captured.medi }\OperatorTok{+}\StringTok{ }\NormalTok{check.medi}
\NormalTok{\}}
\NormalTok{final <-}\StringTok{ }\KeywordTok{mean}\NormalTok{(captured.medi)}\OperatorTok{/}\NormalTok{sims.evaluation}
\NormalTok{final}
\end{Highlighting}
\end{Shaded}

\begin{verbatim}
## [1] 0.986
\end{verbatim}

\textbf{Step:} Generate a single sample from a standard normal
distribution of size \emph{N} = 201. Explain to the reader how you use
MLE to estimate the distribution.

Used MLE to take the mean and standard deviation of the 201 samples. We
can do this because the samples are random on a normal distribution.

\textbf{Step:} Show the reader how you approximate the sampling
distribution of the median, conditional on the estimate of the
distribution in the previous step.

we use the samples' means generated in previous step to determine median
to determine the approximate sampling distribution.

\textbf{Step:} Describe how you calculate a 95\% confidence interval
from the approximated sampling distribution.

Using the stored approximated sampeling distributions and and find the
middle 95\% (from 2.5\% to 97.5\%).

\textbf{Step:} Explain the concept of coverage probability. Explain your
code for calculating the coverage probability.

Next we check how many of the confidence intervals at 95\% percent
contain the true population mean (0). We check that the bottom 2.5\%
interval is less than the population mean and that the top 97.5\%
interval is greater than zero. If so the spread covered - if not it
hasen't covered.

\textbf{Step:} Perform the simulation and report the results.

We do the steps above many times (5000) to generate many coverages - we
then divide the ones that covered by the total number of sims. This will
give us the proportion of simulations whose confidence interval covered
the population mean. In one run I had 4925 of the 5000 simulations cover
- meaning we have a coverage probability of 98.5\%!

\textbf{Step:} Describe how you might change the simulation to learn
more about the operating characteristics of your chosen method for
constructing the 95\% confidence interval.

We could make the population standard deviation tighter to see if
hitting the cover becomes easier. We could also try to look at not the
middle 95\% around the median - but the tightest possible 95\%
confidence interval in each simulation. This should be centered around
the median as we used a normal distribution but it will be different
simulation to simulation.


\end{document}
